\documentclass{article}
\usepackage{graphicx} % Required for inserting images

\title{\textbf{Exploración de escenarios de información escasa en la gestión del caudal ambiental: un enfoque bayesiano para cuantificar la incertidumbre en datos de caudal}}
\author{Omar Alfredo Mercado Díaz}
\date{March 2024}

\begin{document}

\maketitle
{Examinamos el impacto de la incertidumbre en las mediciones del caudal para las estimaciones del caudal ambiental, explorando escenarios de información escasa. Para esto se realiza la cuantificación de la incertidumbre en datos diarios de caudal a través de un modelo basado en el enfoque bayesiano, reconociendo y abordando las limitaciones de la información disponible. La incertidumbre en los modelos se debe principalmente a cambios en las condiciones del entorno natural, limitaciones en la instrumentación, errores en las mediciones y falta de datos (Beck, 1987). 
Generalmente las investigaciones abordan tres tipos de incertidumbres cuando de analizar la incertidumbre en modelos ambientales (Kiang, 2018) (Deletic et al., 2009, Freni et al., 2008, Willems, 2008 y Lindenschmidt et al., 2007).
Incertidumbre estructural: incertidumbre debida a la estructura del modelo, que incluye las ecuaciones y algoritmos utilizados para las simulaciones y acoplamiento de modelos.
Incertidumbre de los datos de entrada: incertidumbre en los datos utilizados como condiciones de contorno e iniciales en el modelo.
Incertidumbre de los parámetros : incertidumbres en los parámetros del modelo e información incierta utilizada para la estimación de parámetros.
Los atributos del régimen de caudal tienen particular influencia en la salud de los ecosistemas acuáticos (magnitud, duración, frecuencia, tasa de cambio, momento de aplicación o de ocurrencia). El régimen de caudal ambiental de referencia dependerá del objetivo ambiental o condición ecológica del cuerpo de agua. Si dicho objetivo corresponde con la prestación de servicios ecosistémicos asociados a soporte o regulación, se deberá caracterizar el régimen natural de flujo, mientras que si dicho objetivo está asociado con servicios ecosistémicos culturales o de aprovisionamiento, la caracterización se deberá realizar considerando la oferta hídrica total en condiciones actuales de alteración y las métricas corresponderán con los usos que se tengan como prioridades partiendo de lo más o lo menos restrictivo. Los enfoques a utilizar dependerán de la información hidrológica disponible y de la calidad de los mismos, razón por la cual la primera actividad a realizar es determinar la incertidumbre de los datos, para el posterior análisis hidrológico con los métodos que apliquen. Una forma muy común de simular el efecto de las incertidumbres en los datos medidos es utilizar distribuciones aleatorias para describir el error en los datos de entrada en un modelo y evitar el uso de valores fijos.
En función del nivel de instrumentación hidrometeorológica disponible en la cuenca hidrográfica, se tendrán diferentes escenarios que condicionarán el método a utilizar para la caracterización del régimen hidrológico. El primer escenario será la disponibilidad de información para un tramo de interés en particular; los datos de una estación para que sean pertinentes deben tener una longitud de registros sistemáticos a escala diaria de al menos 15 años, y no tener dentro de esta longitud más del 10% de datos faltantes. Cumplidos tales criterios, se deberán considerar los respectivos análisis de calidad en la información (homogeneidad, consistencia y detección de datos anómalos). El segundo escenario será la caracterización del régimen hidrológico a lo largo de toda la red drenaje más allá del tramo, teniendo en cuenta, además de las estaciones de caudal, la disponibilidad de estaciones de registro de precipitación las cuales cumplan con los criterios anteriormente mencionados respecto de longitud, datos faltantes y calidad. En los casos que no se cuente con información ni de precipitación ni de caudal en la cuenca hidrográfica objeto de estudio, se deberán aplicar técnicas de estimación en cuencas no aforadas o con escasos datos de aforos. 
En esta investigación abordamos la exploración de modelos bayesianos para estudiar la incertidumbre en escenarios con información limitada de caudal y precipitación de estaciones hidrometeorológicas en Colombia, pertenecientes a la red del Instituto de Hidrología, Meteorología y Estudios Ambientales (IDEAM). La representatividad de las estaciones de registro de caudal emerge como un punto crítico y se presenta en función de la disponibilidad de datos, factores fisiográficos y cobertura espacio-temporal del área de la cuenca; garantizando que las estaciones seleccionadas sean verdaderamente representativas del comportamiento hidrológico.
El propósito de esta metodología apunta hacia la obtención de series de caudal medio diario confiables, de tal forma que sea viable llevar a cabo una caracterización del régimen de caudales ambientales en condiciones no alteradas o de referencia. Para esto, la literatura propone como condición responder a varios interrogantes (Kiang, 2018), en este trabajo intentaremos responder las siguientes: 
¿Cuántas mediciones se requieren para las estimaciones de incertidumbre en caudales y cómo deberían distribuirse en todo el rango de flujo?
¿Cómo se deben manejar los datos atípicos y cuestionables (o más o menos ciertos)?
No existe un método único que permita optimizar la estimación de la incertidumbre en los datos de caudal, debido a que cada método hace suposiciones diferentes sobre las fuentes y los tipos de incertidumbre. La metodología propuesta utiliza un modelo centrado en Bayes, lo que permite la determinación confiable de datos de caudal y la cuantificación de la incertidumbre inherente en estos datos. El modelo reconoce y aborda las limitaciones de la información disponible, proporcionando una herramienta valiosa para gestionar la incertidumbre, un factor clave en la toma de decisiones relacionadas con los recursos hídricos.
Este trabajo pretende contribuir a mejora en la estimación del caudal ambiental a través de la implementación de un modelo bayesiano que reduce las incertidumbres en los datos de caudal y proporciona información confiable para la toma de decisiones, abriendo nuevas perspectivas para una gestión más eficiente y sostenible de los recursos hídricos.
}
\end{document}
